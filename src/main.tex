
\documentclass[12pt]{article}
\usepackage{lmodern}
\usepackage[margin=0.5in]{geometry}

\title{PLCC Plugin for Intellij}
\author{Mark Nash (mdn4993)}
\date{October 2020}

\begin{document}
{\fontfamily{lmss}\selectfont
\maketitle

\begin{abstract}
    PLCC\cite{plcc-paper} is a custom compiler compiler tool used to teach the Programing Language Concepts course at Rochester Institute of Technology.
    Taking a course would be less stressful and give more time for students to learn concepts if students could fight less with the custom tool and just worry about their assignments.
    Intellij is a popular IDE that exposes an SDK to develop extra functionality on it.
    RIT students are farmilliar with this IDE since it is prefered for two pre-requisite classes.
    The current development process for an assignment in the PLC class requires writing a file without syntax highlighting, then compiling the file, then running the generated compiler.
    Syntactic and symantic errors could exist in the file and cause pain to debug.
    The different ways to start a generated compiler could also be confusing.
    The Intellij platform contains SDK functionality to ease these development cycle painpoints and more.
\end{abstract}

\section{Introduction}
The current development process for using the tool is to first manually download it (if not using the RIT CS servers which has it pre-installed), then write a PLCC file containing the grammar rules, regexes for tokens, and extra java files for custom functionality, then run a command to generate the language interpreter, then run either a repl or an execution or a trace execution.
These executions have different command names and it is difficult to remember which ones do what.
Currently no formal effort has been done to make this process easier.
Intellij has the functionality to automatically download files, so this is able to be done if a user wants PLCC to be downloaded on their home computer.
Intellij also has the idea of a run configuration that is a graphical setup of what can esentially be boiled down to a command-line command.
The creater of the plugin just defines what GUI functionality corresponds to what command/argument is ran.
This can help to guide the user which commands to run by adding documentation to the GUI before a command is run.
Intellij syntax highlighting can help users see syntax errors, and common semantic errors can be caught by additions to the AST Intellij provides.

\section{Project Creation}
Project creation consists of two main things: PLCC tool detection and the creation of a standard project file layout.
When a user clicks File -> New Project, they should be presented with an option on the left that says PLCC.
Clicking on PLCC will then let them click through a project creation wizard whose first step is finding the PLCC instalation.
Clicking 'next' will let them pick a directory for the project to be located.
Clicking finish will give them an empty project.

\section{}

\bibliographystyle{ACM-Reference-Format}
\bibliography{plcc_bib}


}
\end{document}