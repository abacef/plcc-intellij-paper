
\documentclass[12pt]{article}
\usepackage{lmodern}
\usepackage[margin=0.5in]{geometry}

\title{PLCC Plugin for Intellij}
\author{Mark Nash (mdn4993)}
\date{October 2020}

\begin{document}
{\fontfamily{lmss}\selectfont
\maketitle

\begin{abstract}
    PLCC\cite{plcc-paper} is a custom compiler compiler tool used to teach the Programing Language Concepts course at Rochester Institute of Technology.
    Taking a course would be less stressful and give more time for students to learn concepts if students could fight less with the custom tool and just worry about their assignments.
    Intellij is a popular IDE that exposes an SDK to develop extra functionality on it.
    RIT students are farmilliar with this IDE since it is prefered for two pre-requisite classes.
    The current development process for an assignment in the PLC class requires writing a file without syntax highlighting, then compiling the file, then running the generated compiler.
    Syntactic and symantic errors could exist in the file and cause pain to debug.
    The different ways to start a generated compiler could also be confusing.
    The Intellij platform contains SDK functionality to ease these development cycle painpoints and more.
\end{abstract}

\section{Introduction}
The current development process for using the tool is to first manually download it (if not using the RIT CS servers which has it pre-installed), then write a PLCC file containing the grammar rules, regexes for tokens, and extra java files for custom functionality, then run a command to generate the language interpreter, then run either a repl or an execution or a trace execution.
These executions have different command names and it is difficult to remember which ones do what.
Currently no formal effort has been done to make this process easier.
Intellij has the functionality to automatically download files, so this is able to be done if a user wants PLCC to be downloaded on their home computer.
Intellij also has the idea of a run configuration that is a graphical setup of what can esentially be boiled down to a command-line command.
The creater of the plugin just defines what GUI functionality corresponds to what command/argument is ran.
This can help to guide the user which commands to run by adding documentation to the GUI before a command is run.
Intellij syntax highlighting can help users see syntax errors, and common semantic errors can be caught by additions to the AST Intellij provides.

\section{Project Creation}
Project creation consists of two main things: PLCC tool detection and the creation of a standard project file layout.
When a user clicks File -> New Project, they should be presented with an option on the left that says PLCC.
Clicking on PLCC will then let them click through a project creation wizard whose first step is finding the PLCC instalation.
Clicking 'next' will let them pick a directory for the project to be located.
Clicking finish will give them an empty project.

\section{Run Configurations}
When a user has a PLCC project open they will be able to add a PLCC run configuration.
In the configuration pane, there will be a combo box with the four commands that can be run on a generated parser.
The tool will automatically compile the .plcc file into its java files so that a user does not forget to re-compile when they have made a change to the file.
When run, a run configuration will execute the specified command on the integrated terminal.

\section{Commenting Command}
The Intellij platform lets a user highlight lines then press Ctrl + / which makes the lines be commented out.
This takes less time than commenting a line one by one.
The comment character in a .plcc file is \# which is an EOL comment.
This character will be appended to each line highlighted when pressing Ctrl + / in a .plcc file.

\section{Syntax Highlighting}
It is helpful to see where syntax elements start and end while programing.
When writing a .plcc file, syntax elements are highlighted in a different color with each color similarly corresponding to Java syntax elements.
The specific color is not set in stone, but the color is a generic Intellij platform type color that will be able to change to match the IDE theme selected.
If some syntax is not correct, the incorrect characters are underlined in a red squiggle letting the user know that the syntax is invalid.

\section{References and Name Resolution}
Sometimes it is nice to know where a syntax element is defined or used previously.
When a user Ctrl + clicks on some syntax, if that element has usages, a list will pop up showing where specifically they are in the file.
If the element is a usage, a Ctrl + click will move the cursor to its definition.
This functionality is implemented for regex names/usages, and grammar element names/usages.

\section{User Testing}
This prototype plugin was given to three volunteers currently taking the PLCC course and they generally said that they (liked/disliked) the tool and they thought it was (helpful/unhelpful).
The following elements of the plugin were enjoyed while the following elements could use a little work. The following elements were not useful.
They also requested the following features.

\section{Conclusion}
Given our generally (positive/negative) results from user testing, it may be useful in the future to (continue adding features to this project like: /integrate the tool as it is now into the curriculum/Not continue development of the plugin because it was more of a pain than the previous way of using PLCC).

\bibliographystyle{ACM-Reference-Format}
\bibliography{plcc_bib}


}
\end{document}